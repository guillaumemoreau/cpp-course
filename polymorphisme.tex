% % polymorphisme

\begin{frame}{Rappels : les classes}
\begin{itemize}
	\item La réutilisation est un aspect important de l'héritage, mais pas forcément le plus important !
	\item Le deuxième point fondamental est la relation qui lie une classe à sa super-classe :
	\begin{itemize}
		\item une classe $B$ qui hérite de la classe $A$ peut être vue comme un \textbf{sous-type} (sous-ensemble) du type défini par la classe $A$
	\end{itemize}
\end{itemize}
%% insérer ici un exemple : un étudiant sportif est une sorte d'étudiant
\end{frame}

\subsection{Surclassement}

 %%%%%%%%%%%%%%%%%%%%%%%%%%%%%%%%%%%%%%%%%%%%%%
\begin{frame}[fragile]\frametitle{Surclassement}
\begin{itemize}
	\item Tout objet instance de la classe $B$ peut aussi être vu comme une instance de la classe $A$
	\item Cette relation est directement supportée par le langage C++
	\begin{itemize}
		\item à une référence de type $A$, il est possible d'affecter une valeur qui est une référence vers un objet de type $B$ (surclassement ou upcasting)
		\item plus généralement, à une référence d'un type donné, il est possible d'affecter une valeur qui correspond à une référence dont le type effectif est n'importe quelle sous-classe du type de la référence
	\end{itemize}
	\item Exemple : en supposant définie une classe \textit{Etudiant} dérivant d'une classe \textit{Personne}, on peut écrire
	\begin{lstlisting}
	Personne *p = new Etudiant();
	\end{lstlisting}
	\item Pourquoi utiliser des pointeurs ici ?
\end{itemize}
\end{frame}

\begin{frame}[fragile]
\frametitle{Le problème}
\begin{itemize}
\item Déclaration d'une fonction d'affichage
\begin{lstlisting}
void afficher(exheritageA a);
\end{lstlisting}
\item Définition de la fonction
\begin{lstlisting}
void afficher(exheritageA a) {
    a.affiche();
}
\end{lstlisting}
\item Utilisation
\begin{lstlisting}
    afficher(a);
    afficher(b);
    afficher(c);
\end{lstlisting}
\pause\begin{block}{Résultat}
{\tiny
\begin{verbatim}
je suis un objet de la classe exheritageA
je suis un objet de la classe exheritageA
je suis un objet de la classe exheritageA
\end{verbatim}
}
\end{block}
\pause
\item Explication : passage par valeur $\Longrightarrow$ copie de l'objet
\end{itemize}
\end{frame}

\begin{frame}[fragile]
\frametitle{Le problème (suite)}
\begin{itemize}
\item Passons par un pointeur !
\item Déclaration de la nouvelle fonction
\begin{lstlisting}
void afficherPtr(exheritageA *a);
\end{lstlisting}
\item Définition de la fonction
\begin{lstlisting}
void afficherPtr(exheritageA *a) {
    a->affiche();
}
\end{lstlisting}
\item Utilisation
\begin{lstlisting}
    afficherPtr(&a);
    afficherPtr(&b);
    afficherPtr(&c);
\end{lstlisting}
\pause\begin{block}{Résultat}
{\tiny
\begin{verbatim}
je suis un objet de la classe exheritageA
je suis un objet de la classe exheritageA
je suis un objet de la classe exheritageA
\end{verbatim}
}
\end{block}
\pause
\item Insuffisant (en C++) : notion de fonction \textit{virtuelle} nécessaire
\end{itemize}
\end{frame}

\begin{frame}{Fonction virtuelle}

\begin{itemize}
\itemsep1pt\parskip0pt\parsep0pt
\item
  Syntaxe très simple

  \begin{itemize}
  \itemsep1pt\parskip0pt\parsep0pt
  \item
    il suffit d'ajouter le préfixe \emph{virtual} dans la
    \textbf{déclaration} de la classe mère
  \item
    pas obligatoire dans les classes filles et les méthodes redéfinies

    \begin{itemize}
    \itemsep1pt\parskip0pt\parsep0pt
    \item
      mais cela ne fait pas de mal à la compréhension du code
    \end{itemize}
  \item
    NB : pour ceux qui connaissent java, toutes les fonctions sont
    virtuelles en java
  \end{itemize}
\item
  Signification

  \begin{itemize}
  \itemsep1pt\parskip0pt\parsep0pt
  \item
    La résolution \emph{statique} des liens est remplacée par une
    résolution \emph{dynamique}
  \item
    Si on utilise des fonctions virtuelles \textbf{et} des
    pointeurs/références vers un objet, alors la méthode appelée sera
    celle du type réel de l'objet
  \end{itemize}
\item
  Ca sera plus clair sur un exemple !

  \begin{itemize}
  \itemsep1pt\parskip0pt\parsep0pt
  \item
    Reprise de l'exemple précédent en rendant affiche() virtuelle
  \end{itemize}
\end{itemize}

\end{frame}

\begin{frame}[fragile]
\frametitle{Exemple simpliste}
\begin{itemize}
\item Seule modification : la déclaration de \textit{affiche()} dans \textit{exheritageA}
\begin{lstlisting}
    virtual void affiche();
\end{lstlisting}
\item Utilisation inchangée
\begin{lstlisting}
    afficher(a);
    afficher(b);
    afficher(c);
    
    afficherPtr(&a);
    afficherPtr(&b);
    afficherPtr(&c);
\end{lstlisting}
\pause
{\tiny
\begin{verbatim}
je suis un objet de la classe exheritageA
je suis un objet de la classe exheritageA
je suis un objet de la classe exheritageA
je suis un objet de la classe exheritageA
je suis un objet de la classe exheritageB
je suis un objet de la classe exheritageA
je suis un objet de la classe exheritageC
\end{verbatim}
}
\end{itemize}
\end{frame}

\begin{frame}{Surclassement bis}
\begin{itemize}
	\item Lorsqu'un objet est surclassé, il est vu comme un objet du type de la référence utilisée pour le désigner
	\item Ses fonctionnalités sont alors limitées à celle du type de la référence
	\item Résolution des messages :
	\begin{itemize}
		\item Que se passe-t-il si on appelle une méthode (virtuelle) surchargée ?
		\item c'est bien la méthode du type <<réel>> de la référence qui est appelée, telle qu'elle est définie au niveau de la classe
	\end{itemize}
\end{itemize}
\end{frame}

\begin{frame}[fragile,containsverbatim]
\frametitle{Exemple : limitation de fonctionnalités}
\begin{itemize}
\item on ajoute à \emph{exheritageC}
\begin{lstlisting}
public void methodeSupplementaire() {
  // ne fait rien
}
\end{lstlisting}
\item dans le \texttt{main()} :
\end{itemize}
\begin{lstlisting}
    exheritageA *ptrC = &c;
    ptrC->affiche();
    ptrC->methodeSupplementaire();
\end{lstlisting}
\begin{block}{Exécution}
{\tiny \begin{verbatim}
exheritage.cxx:59:11: error: no member named 'methodeSupplementaire' in 'exheritageA'
    ptrC->methodeSupplementaire();
    ~~~~  ^
1 error generated.
\end{verbatim}}
\end{block}
\end{frame}

\begin{frame}{Liaison dynamique (uniquement méthodes \textit{virtuelles})}
\begin{itemize}
	\item Les messages sont résolus à l'exécution
	\begin{itemize}
		\item La méthode exécutée est déterminée à l'exécution (run-time) et non pas à la compilation
		\item à cet instant (et seulement à cet instant), le type exact de l'objet qui reçoit le message est connu
		\item ce mécanisme est connu sous le nom de \textbf{liaison dynamique} ou dynamic binding ou encore late-binding voire run-time binding
	\end{itemize}
	\item A la compilation, seules les vérifications statiques sont effectuées (signature)
\end{itemize}
\end{frame}

\begin{frame}[fragile]\frametitle{Vérification statique insuffisante}
\begin{itemize}
	\item à la compilation il n'est \alrt{pas possible} de déterminer le type exact de l'objet qui reçoit le message
	\item la vérification statique garantit simplement dès la compilation que les messages pourront être résolus
\end{itemize}
\begin{codeblock}{La preuve !}
\begin{lstlisting}
    exheritageA *ex[100];
    for (int i=0 ; i<100 ; i++) {
        int z = rand() % 100;
        if (z > 50) {
            ex[i] = new exheritageB();
            
        }
        else {
            ex[i] = new exheritageC();
        }
        ex[i]->affiche();
    }
\end{lstlisting}
\end{codeblock}
\end{frame}

\begin{frame}{Cas particulier : constructeurs et destructeurs}
\begin{itemize}
\item Les constructeurs ne peuvent pas être virtuels : 
\begin{itemize}
\item à la construction, on sait quel objet on va construire !
\item il est également interdit de faire appel à une méthode virtuelle dans un constructeur
\end{itemize}
\item Le destructeur peut être rendu virtuel
\begin{itemize}
\item voire : le destructeur \alrt{devrait} être virtuel (lorsqu'on utilise le polymorphisme)
\item sinon appel à \texttt{delete} sur un pointeur surclassé génère un appel au mauvais destructeur !
\end{itemize}
\end{itemize}
\end{frame}

\subsubsection{Exemple : comptes rémunérés}

\begin{frame}{Exemple : comptes rémunérés}
\begin{itemize}
\item Il existe des comptes bancaires auxquels les banques apportent une rémunération
\begin{itemize}
\item Par exemple, les livrets bancaires bénéficient d'une rémunération annuelle\footnote{calculée tous les 15j, mais non considéré ici}
\item Mais il existe aussi des comptes courants rémunérés qui bénéficient d'intérêts pour lorsque le solde d'un compte dépasse un certain seuil
\end{itemize}
\item Point commun : ils ont tous une fonction de calcul de rémunération du compte
\item Souhait de l'agence : Pour chaque compte, appeler la fonction de rémunération du compte
\end{itemize}
\end{frame}

\begin{frame}{Architecture de la solution}
\begin{itemize}
\item Une classe mère commune à tous les comptes rémunérés (qui dérive elle-même de \textit{compte})
\begin{itemize}
\item Elle comporte une méthode de calcul des intérêts qui ne fait rien
\end{itemize}
\item Une classe \textit{livret} qui hérite de la classe précédente et qui implémente sa propre version du calcul d'intérêts
\item Une classe \textit{compte courant rémunéré} qui fait de même
\item Une agence bancaire contient une liste de comptes rémunérés (de tous types)
\begin{itemize}
\item polymorphisme : pour chaque compte rémunéré, on peut appeler la méthode de calcul appropriée
\item C++ : nécessité d'utiliser des pointeurs dans la liste et des méthodes virtuelles
\end{itemize}
\item pour aller plus loin : \hyperlink{secabstract}{\beamergotobutton{classes abstraites}}, \hyperlink{sectypeid}{\beamergotobutton{introspection}}, \hyperlink{secstl}{\beamergotobutton{librairie standard}}
\end{itemize}
% @TODO1908 : corriger les liens vers les sections
\end{frame}

\begin{frame}[fragile]
\frametitle{Exemple : déclaration des classes}
\begin{codeblock}{Classe \textit{compteremun}}
\begin{lstlisting}
class compteremun : public compte {
public:
    virtual void calc_interets();
    compteremun(int n);
};
\end{lstlisting}
\end{codeblock}

\begin{codeblock}{Classes \textit{livret} et \textit{ccourantremun}}
\begin{lstlisting}
class livret : public compteremun {
protected:
    float tauxannuel;
public:
    virtual void calc_interets();   
    livret(int n,float t);
};

class ccourantremun : public compteremun {
protected:
    float tauxquotidien;
    float seuil;
public:
    virtual void calc_interets();   
    ccourantremun(int n,float t,float s);
};
\end{lstlisting}
\end{codeblock}

\end{frame}

\begin{frame}[fragile]\frametitle{Exemple : définition des méthodes}
\begin{lstlisting}
void compteremun::calc_interets() {
    // do noting
}

compteremun::compteremun(int n) : compte(n) {
    // do nothing
}

void livret::calc_interets() {
    solde += solde*(1.0+tauxannuel);
}

livret::livret(int n,float t) : compteremun(n) {
    tauxannuel = t;
}

ccourantremun::ccourantremun(int n,float t,float s) : compteremun(n) {
    tauxquotidien = t;
    seuil = s;
}

void ccourantremun::calc_interets() {
    if (solde > seuil) {
        solde += (solde-seuil)*(1.0+tauxquotidien);
    }
}
\end{lstlisting}
\end{frame}

\begin{frame}[fragile]\frametitle{Exemple : calcul des intérêts}
\begin{codeblock}{Liste de comptes}
\begin{lstlisting}
    list<compteremun*> agence;
    agence.push_back(new livret(1,0.03));
    agence.push_back(new ccourantremun(1,0.001,1500.0));
\end{lstlisting}
\end{codeblock}

\begin{codeblock}{Calcul des intérêts}
\begin{lstlisting}
    for (list<compteremun*>::iterator i=agence.begin() ; i != agence.end(); i++) {
        (*i)->calc_interets();
    }
\end{lstlisting}
\end{codeblock}

\begin{itemize}
\item \alert{Problème} : la méthode \verb!compteremun::calc_interets()! qui ne fait rien, on aimerait
\begin{itemize}
\item Forcer la redéfinition d'une méthode dans la classe fille et interdire l'instanciation de \textit{compteremun}
\end{itemize}
\item \structure{Solution} : les classes et méthodes abstraites
\end{itemize}
\end{frame}