%%%%%%%%%%%%%%%%%%%%%%%%%%%%%%%%%%%%%%%%%%%%%%%%%%%%%%%%%%%%%%%%%%%%%%%%%%%%
%%                                                                        %%
%%                 Support de cours Programmation orientée objet                  %%
%%                                                                        %%
%%%%%%%%%%%%%%%%%%%%%%%%%%%%%%%%%%%%%%%%%%%%%%%%%%%%%%%%%%%%%%%%%%%%%%%%%%%%
%% Guillaume Moreau (EC Nantes)
%% création : 21/06/2004
%% dernière modification : 23/06/2004
%% historique :

%% il faut fixer l'URL base pour que les liens relatifs fonctionnent...
%% bizaremment fichu mais c'est comme ça.

\documentclass[allowframebreaks,xcolor=dvipsnames]{beamer}

\mode<presentation>
{
\usetheme{Antibes}
  % or ...

  %\setbeamercovered{transparent}
  % or whatever (possibly just delete it)
}
\useoutertheme{infolines}
\usecolortheme[named=RoyalBlue]{structure}

\uselanguage{french}
\languagepath{french}
\deftranslation[to=french]{definition}{définition}
\deftranslation[to=french]{Definition}{D\'efinition}

\setbeamertemplate{blocks}[rounded][shadow=true]
\setbeamertemplate{navigation symbols}{}
\setbeamertemplate{itemize item}[square]
\setbeamertemplate{itemize subitem}[triangle]

\usepackage[T1]{fontenc}
\usepackage[utf8]{inputenc}
% or whatever

\usepackage[french]{babel}
% or whatever

%% pour afficher le plan à chaque début de section
%\AtBeginSection[]{
%  \begin{frame}{Plan}
%  \tiny \tableofcontents[currentsection, hideothersubsections]
%  \end{frame}
%}


%% tout ce qui est relatifs aux extraits de code
\usepackage{listings}
\usepackage{lstautogobble}
\lstloadlanguages{C++}
\lstset{% paramètres généraux des listings
	language=C++,
	basicstyle=\ttfamily\tiny, %% style général : chasse fixe, taille minimale
	keywordstyle=\color{webgreen}\bfseries, % les mots clés en vert et gras
	stringstyle=\color{blue}, % les chaines de caractères en bleu
	commentstyle=\color{webbrown},
	autogobble
}


\usepackage{myslides}

%\usepackage{times}
\usepackage[T1]{fontenc}
% Or whatever. Note that the encoding and the font should match. If T1
% does not look nice, try deleting the line with the fontenc.


\title[Option RV / CPLUS] % (optional, use only with long paper titles)
{Programmation C++}

\author[G. Moreau]{Guillaume Moreau\\
\texttt{guillaume.moreau@ec-nantes.fr}}
% - Use the \inst{?} command only if the authors have different
%   affiliation.

\institute[Ecole Centrale de Nantes] % (optional, but mostly needed)
{
  Ecole Centrale de Nantes
}

\date % (optional)
{Septembre 2020}

\subject{Cours Programmation C++}


\definecolor{webgreen}{rgb}{0,.5,0}
\definecolor{webbrown}{rgb}{.6,0,0}
\definecolor{* }{rgb}{0,.5,0}
\definecolor{. }{rgb}{.6,0,0}

% option d'affiche de table des matières : http://mirror.unl.edu/ctan/macros/latex/contrib/beamer/doc/beameruserguide.pdf
% section 10.5

% Delete this, if you do not want the table of contents to pop up at
% the beginning of each subsection:
\AtBeginSection[]
{
   \begin{frame}
       \frametitle{Plan}
       \tableofcontents[sectionstyle=show/hide,subsectionstyle=show/show/hide,subsubsectionstyle=hide/hide]
   \end{frame}
}

\AtBeginSubsection[]
{
	\begin{frame}
		\frametitle{Plan}
		\tableofcontents[sectionstyle=show/hide,subsectionstyle=show/shaded/hide,subsubsectionstyle=show/show/hide]
	\end{frame}
}

\begin{document}

\begin{frame}
  \titlepage
\end{frame}

\begin{frame}[allowframebreaks]{Plan du cours}
  \tableofcontents[hideallsubsections]
  % You might wish to add the option [pausesections]
\end{frame}



\section{La bibliothèque standard}

% la STL

\begin{frame}{Introduction}
\label{secstl}
\begin{itemize}
\item \hypertarget{pdf:stl}{Un} langage de programmation a besoin d'une bibliothèque d'interaction avec le système
\begin{itemize}
\item On ne peut pas se contenter de faire tourner des algorithmes en rond sans entrées/sorties
\item Et même pour écrire des algorithmes, on ne veut pas (mal) réinventer la roue à chaque fois
\end{itemize}
\item Les fonctions livrées avec C++ ont longtemps été le parent pauvre du langage (notamment par rapport à Java)
\item Pire, dans les anciennes versions C++ (i.e. avant la normalisation), on trouvait des bibliothèques (très) variables
\item On s'intéresse à la version normalisée C++11 de la \textit{SL}
\begin{itemize}
\item pas de doc officielle d'utilisation, juste une description fonctionnelle dans la norme...
\item Néanmoins, très bonne doc : \url{http://www.cplusplus.com/reference/}
\end{itemize}
\end{itemize}
\end{frame}

\begin{frame}[fragile]\frametitle{Contenu}
\begin{itemize}
\item 3 grandes parties dans la \textit{SL}
\begin{itemize}
\item Un héritage du C (nécessaire)
\item La STL (\textit{Standard Template Library}) et les flots
\item Le reste (dont les \verb|string|, les \verb|namespace|)...
\end{itemize}
\end{itemize}
\end{frame}

\subsection{L'héritage du C}

\begin{frame}[fragile]\frametitle{L'héritage du C}
\begin{itemize}
\item En C, on avait pris l'habitude de faire des \verb|#include <math.h>|
\item En C++, on les remplace progressivement par \verb|#include <cmath>|
\item On y ajoute parfois un \verb|using namespace std;|
\begin{itemize}
\item permet d'écrire \verb|sqrt(2)| au lieu de \verb|std::sqrt(2.0)|
\end{itemize}
\item Tout ce qui était utilisé en C est encore disponible sous cette nouvelle dénomination, mais pas forcément utile
\end{itemize}
\end{frame}

\subsection{La STL}

\begin{frame}{La STL}
\begin{itemize}
\item STL : Standard Template Library
\item On parle aussi de bibliothèques de conteneurs (parce les structures fournies ne sont seront pas liées à un type donné)
\item Correspond au \textit{Collection Framework} de java
\begin{itemize}
\item en beaucoup plus puissant
\item et peut s'avérer beaucoup plus compliqué également...
\end{itemize}
\item Commençons par un exemple : le vecteur
\begin{itemize}
\item Alternative pratique aux tableaux standards du C
\end{itemize}
\end{itemize}
\end{frame}

\begin{frame}[fragile]\frametitle{Le conteneur \texttt{vector}}
\begin{itemize}
\item Les listes sont une des abstractions les plus utiles en programmation
\begin{itemize}
\item Que faire avec ? créer, parcourir, rechercher, ajouter, supprimer...
\end{itemize}
\item Toujours les mêmes algorithmes quel que soit le type d'éléments de la liste
\item Structure et algorithmes universels : réutiliser ce qui a déjà été (bien) écrit (et testé)
\item C++ fournit la structure \verb|vector<type>| où \verb|type| est à remplacer par un type de données
\begin{itemize}
\item  \verb|vector<int>| pour un vecteur d'entiers
\item \verb|vector<forme>| pour un vecteur de formes
\item \verb|vector<forme*>| pour un vecteur de pointeurs sur des formes
\item \verb|vector<vector<float> >| pour une vecteur de vecteurs de réels
\end{itemize}
\end{itemize}
\end{frame}

\begin{frame}[fragile]\frametitle{Exemple d'utilisation de \texttt{vector}}
\begin{lstlisting}
#include <vector>
#include <iostream>
using namespace std;

int main() {
    vector<int> v1;
    v1.push_back(5);
    v1.push_back(8);
    for (int i(0) ; i<8 ; i++) {
        v1.push_back(i);
    }
    cout << "taille : " << v1.size() << endl;

    for (int i(0) ; i<v1.size() ; i++) {
        cout << v1[i] << endl;
    }

    // syntaxe alternative
    for (int& x : v1) {
        cout << x << endl;

    }
}
\end{lstlisting}
\end{frame}

\begin{frame}[fragile]\frametitle{Conteneurs pour les structures linéaires}
\begin{itemize}
\item Rappel : suivant la structure interne utilisée (liste chaînée ou tableau par exemple), une structure linéaire est plus ou moins efficace selon les opérations
\item C++ met à disposition 2 grands types de structures linéaires qui s'utilisent en grande partie de façon identique
\begin{itemize}
\item Les séquences
\begin{itemize}
\item \verb|vector| \urldoc{http://www.cplusplus.com/reference/vector/vector/} : tableau classique
\item \verb|deque| \urldoc{http://www.cplusplus.com/reference/deque/deque/} : liste doublement chaînée (\textit{double ended queue})
\item \verb|list| \urldoc{http://www.cplusplus.com/reference/list/list/} : conteneur générique pour l'utilisation des itérateurs
\item \verb|stack| \urldoc{http://www.cplusplus.com/reference/stack/} : pile
\item \verb|queue| \urldoc{http://www.cplusplus.com/reference/queue/} : file (structure FIFO)
\item \verb|priority_queue| \urldoc{http://www.cplusplus.com/reference/queue/priority_queue/} : file avec priorité (suppose des éléments comparables)
\end{itemize}
\item Les tableaux associatifs
\begin{itemize}
\item \texttt{set} \urldoc{http://www.cplusplus.com/reference/set/set/}
\item \texttt{multiset} \urldoc{http://www.cplusplus.com/reference/set/multiset/}
\item \texttt{map} \urldoc{http://www.cplusplus.com/reference/map/map/}
\item \texttt{multimap} \urldoc{http://www.cplusplus.com/reference/map/multimap/}
\end{itemize}
\end{itemize}
\end{itemize}
\end{frame}

\begin{frame}[fragile]\frametitle{Utilisation la plus commune possible}
\begin{itemize}
\item C++ essaie au maximum de faciliter l'utilisation des conteneurs
\begin{itemize}
\item inclusion : \verb|#include <conteneur>|
\item quelques méthodes communes
\begin{itemize}
\item \verb|empty()|
\item \verb|size()|
\item \verb|clear()|
\end{itemize}
\item Utilisation maximale de la surcharge des opérateurs (vu pour l'accès à des éléments d'un vecteur)
\end{itemize}
\end{itemize}
\end{frame}

\subsubsection{Les séquences}

\begin{frame}[fragile]\frametitle{Retour sur les vecteurs}
\begin{itemize}
\item Documentation : \url{http://www.cplusplus.com/reference/vector/vector/}
\item Accès aux éléments
\begin{itemize}
\item \verb|operator[]()| pour les accès type tableau
\item \verb|push_back()| pour ajouter à la fin
\item \verb|pop_back()| pour récupérer le dernier élément et l'enlever
\item \verb|front()| premier élément
\item \verb|back()| dernier élément
\end{itemize}
\item En réalité, ces méthodes sont communes à toutes les classes de séquence
\end{itemize}
\end{frame}

\subsubsection{Les tableaux associatifs}

\begin{frame}{Les tableaux associatifs}
\begin{itemize}
\item Dans un \texttt{vector}, les éléments sont accessibles via l'opérateur \texttt{[]} et leur indice qui est un nombre entier
\item On n'a pas toujours besoin d'un indice entier (exemple : dictionnaire)
\item Les tableaux associatifs généralisent la notion de tableau en permettant l'utilisation de n'importe quel type comme indice
\begin{itemize}
\item Déclaration : \texttt{map<}\textit{clé}\texttt{,}\textit{valeur}\texttt{>} \urldoc{http://www.cplusplus.com/reference/map/map/}
\item Le type de clé doit être ordonné
\end{itemize}
\item On parle d'association (\textit{map} en anglais) car à une clé  est associée une valeur \textbf{unique}
\item C'est une notion très courante dans les langages du web comme PhP
\item Les méthodes d'accès sont en ${\cal O}(\log n)$
\end{itemize}
\end{frame}

\begin{frame}[fragile]\frametitle{Utilisation des \texttt{map}}
\begin{itemize}
\item \verb|m.count(key)| : renvoie 1 si la clé est présente, 0 sinon
\item \verb|m.find(key)| : renvoie un itérateur vers la clé ou \verb|m.end()| sinon
\item \verb|m.at(key)| : renvoie la valeur associée  ou exception \verb|out_of_range|
\item \verb|m[key]| : renvoie la valeur associée ou une référence vers une nouvelle valeur
\end{itemize}
\begin{codeblock}{Exemple d'utilisation}
\begin{lstlisting}
    map<string,string> dict;
    dict["banane"] = "banana";
    dict["poire"] = "pear" ;
    dict["pomme"] = "apple";
    cout << (dict.count("cerise")>0 ? "trouve" : "pas trouve") << endl;
    cout << dict.size() << endl;
    cout << dict["framboise"] << endl;
    cout << dict.size() << endl;
\end{lstlisting}
\end{codeblock}
\begin{exampleblock}{Résultat}
{\tiny
\begin{verbatim}
pas trouve
3

4
\end{verbatim}
}
\end{exampleblock}
\end{frame}

\begin{frame}[fragile]\frametitle{Autres conteneurs associatifs}
\begin{itemize}
\item Ils s'utilisent de la même manière
\item Quelques propriétés particulières de chaque type
\begin{itemize}
\item Les \verb|set| \urldoc{http://www.cplusplus.com/reference/set/set/} sont utilisés pour représenter les ensembles, au sens mathématique (insertion et recherche, pas d'accès via []). Fusion valeur/clé en quelque sorte
\item Les \verb|multiset| \urldoc{http://www.cplusplus.com/reference/set/multiset/} sont des ensembles où une valeur peut apparaître plusieurs fois
\item Les \verb|multimap| \urldoc{http://www.cplusplus.com/reference/map/multimap/} sont des \verb|map| où une même clé peut apparaître plusieurs fois
\end{itemize}
\end{itemize}
\end{frame}

\subsubsection{Les itérateurs}

\begin{frame}[fragile]\frametitle{Les itérateurs}
\begin{itemize}
\item Souvenirs de C : le parcours d'une liste se fait en
\begin{itemize}
\item incrémentant un indice $i$ dans le cas d'un tableau
\item en faisant avancer un \textit{pointeur courant} dans le cas d'une liste chaînée
\begin{lstlisting}
ptr = ptr->next;
\end{lstlisting}
\end{itemize}
\item En première approximation, les itérateurs peuvent être vus comme des pointeurs sur les éléments des conteneurs
\item En réalité, ce sont des objets complexes !
\item Analogie avec le pointeur
\begin{itemize}
\item on le fait avancer avec \texttt{++} et reculer avec \texttt{--}
\item on accède à l'élément via l'opérateur de déréférencement \texttt{*}
\end{itemize}
\end{itemize}
\end{frame}

\begin{frame}[fragile]\frametitle{Utilisation}
\begin{itemize}
\item Déclaration : \textit{conteneur}\texttt{::iterator} \textit{nom}\texttt{;}
\begin{lstlisting}
vector<int> tab(5,5);
vector<int>::iterator it;
\end{lstlisting}
\item Itération et utilisation
\begin{lstlisting}
int total = 0;
for (it=tab.begin() ; it!=tab.end() ; it++) {
  total += *it;
}
\end{lstlisting}
\item où \verb|tab.begin()| renvoie un itérateur sur le début de la liste et \verb|tab.end()| un itérateur sur l'élément (invalide) qui suit juste le dernier élément
\item Sur certains types de conteneurs (\verb|vector| et \verb|deque|), l'arithmétique de pointeurs est autorisée (\textit{random access iterators})
\begin{lstlisting}
    *(tab.begin()+3) = 2; // modification du 4eme element
    cout << tab[3] << endl;
\end{lstlisting}
\end{itemize}
\end{frame}

\begin{frame}[fragile]\frametitle{Itérateurs et \texttt{list}}
\begin{itemize}
\item Les \verb|list| sont des structures de listes chaînées dont les éléments sont non contigus en mémoire
\item Les itérateurs constituent le seul moyen d'accès au contenu
\begin{lstlisting}
    list<string> todo;
    todo.push_back("repeindre la cuisine");
    todo.push_back("finir le poly de C++");
    todo.push_back("dormir");

    for (list<string>::iterator i=todo.begin() ; i!=todo.end() ; ++i) {
        cout << "TODO: " << *i << endl;
    }
\end{lstlisting}
\item Syntaxe alternative (et réductrice)
\begin{lstlisting}
    for (string& s : todo) {
        cout << "TODO: " << s << endl;

    }
\end{lstlisting}
\end{itemize}
\end{frame}

\begin{frame}[fragile]\frametitle{Itérateurs et \texttt{map}}
\begin{itemize}
\item C'est un peu plus compliqué puisque les \verb|map| sont des paires $(key,value)$
\item Ca se traduit en C++ par une classe \verb|pair<T,E>| (déclarées dans \verb|utility|) à deux attributs appelés \verb|first| et \verb|second|
\item les itérateurs des \verb|map<T,E>| se déréférencent vers des \verb|pair<K,E>|
\end{itemize}
\begin{codeblock}{Utilisation}
\begin{lstlisting}
    for(map<string,string>::iterator itmap=dict.begin() ; itmap!=dict.end() ; ++itmap) {
        cout << "dict[" << itmap->first << "] = " << itmap->second << endl;
    }
\end{lstlisting}
\end{codeblock}
\begin{block}{Résultat}
{\tiny
\begin{verbatim}
dict[banane] = banana
dict[framboise] =
dict[poire] = pear
dict[pomme] = apple
\end{verbatim}
}
\end{block}
\end{frame}

% foncteurs et algorithmes

\subsection{Foncteurs}

\begin{frame}{Foncteurs}

\begin{itemize}
\itemsep1pt\parskip0pt\parsep0pt
\item
  On aimerait pouvoir appliquer des algorithmes à des collections (au
  sens de la STL)

  \begin{itemize}
  \itemsep1pt\parskip0pt\parsep0pt
  \item
    on peut le faire en créant une fonction qui prend une collection en
    paramètre
  \item
    l'inverse pourrait être pratique : passer la fonction elle-même
    comme paramètre
  \end{itemize}
\item
  C'est le principe des \structure{foncteurs}

  \begin{itemize}
  \itemsep1pt\parskip0pt\parsep0pt
  \item
    On peut aussi utiliser aussi les pointeurs de fonction
  \item
    Un foncteur est une classe qui surcharge l'opérateur \texttt{()} qui effectue
    l'opération souhaitée
  \end{itemize}
  \item Note hors sujet : c'est le principe des \textit{shaders}...
\end{itemize}

\end{frame}

\begin{frame}[fragile]\frametitle{Exemple simpliste}
\begin{itemize}
\item Création du foncteur
\begin{codeblock}{Classe addition}
\begin{lstlisting}
class addition{
public:
    int operator()(int a, int b) {
        return a+b;
    }
};
\end{lstlisting}
\end{codeblock}
\begin{codeblock}{Utilisation}
\begin{lstlisting}
    addition foncteur;
    cout << "2+4 = " << foncteur(2,4) << endl;
\end{lstlisting}
\end{codeblock}
\begin{block}{Résultat}
{\tiny
\begin{verbatim}
2+4 = 6
\end{verbatim}}
\end{block}
\end{itemize}
\end{frame}

\begin{frame}[fragile]\frametitle{Exemple : valeur absolue}
\begin{itemize}
\item Création du foncteur
\begin{codeblock}{Classe addition}
\begin{lstlisting}
class valeurabsolue {
public:
    int operator()(int a) {
        if (a>=0) {
            return a;
        }
        else {
            return -a;
        }
    }
};
\end{lstlisting}
\end{codeblock}
\item On verra par la suite comment l'appliquer à une collection
\end{itemize}
\end{frame}

\begin{frame}[fragile]\frametitle{Plus complexe : remplir une collection avec des valeurs différentes}
\begin{itemize}
\item Les foncteurs sont des objets, ils peuvent avoir des attributs
\item Exemple : un foncteur renvoie une valeur incrémentée à chaque appel
\end{itemize}
\begin{codeblock}{Classe fill}
\begin{lstlisting}
class myfill {
private:
    int m;
public:
    myfill(int i) : m(i) {}
    int operator()() {
        ++m;
        return m;
    }
};
\end{lstlisting}
\end{codeblock}
\begin{codeblock}{Utilisation}
\begin{lstlisting}
    myfill f(22);
    for (vector<int>::iterator i=v1.begin() ; i!=v1.end() ; ++i) {
        *i = f();
    }
\end{lstlisting}
\end{codeblock}
\end{frame}

\begin{frame}{Les prédicats}

\begin{itemize}
\itemsep1pt\parskip0pt\parsep0pt
\item
  les prédicats sont des expressions booléennes (vient de la logique)
\item
  En C++, ce sont des foncteurs particuliers renvoyant un booléen
\item
  Ils sont utilisés pour tester une valeur particulière d'un objet

  \begin{itemize}
  \itemsep1pt\parskip0pt\parsep0pt
  \item
    nombre plus grand que 10
  \item
    cette chaîne contient-elle des voyelles ?
  \item
    Cet objet est-il encore valide ?
  \end{itemize}
\item
  Ils seront très utiles dans les algorithmes de la STL
\end{itemize}

\end{frame}

\begin{frame}[fragile]\frametitle{Les foncteurs prédéfinis}

\begin{itemize}
\itemsep1pt\parskip0pt\parsep0pt
\item
  La STL fournit un certain nombre de foncteurs
\item
  Ils sont déclarés dans \verb|functional| \urldoc{http://www.cplusplus.com/reference/functional/}
  \item Ils sont paramétrés par un type (i.e. on peut choisir le type auquel ils s'appliquent)
\item
  On y trouve notamment

  \begin{itemize}
  \itemsep1pt\parskip0pt\parsep0pt
  \item
    les opérateurs arithmétiques (\verb|plus<T>|)
  \item
    les opérateurs de comparaison (\verb|equal_to<T>|)
  \item
    les opérateurs logiques (\verb|logical_and<T>|)
  \item
    des adaptateurs et des convertisseurs
  \item
    d'autres choses encore...
  \end{itemize}
\end{itemize}

\end{frame}

\subsection{Les algorithmes : lier conteneurs et foncteurs}

\begin{frame}[fragile]\frametitle{Les algorithmes : lien entre conteneurs et foncteurs}

\begin{itemize}
\itemsep1pt\parskip0pt\parsep0pt
\item
  Par algorithme, on entend les fonctions définies dans \verb|algorithm| \urldoc{http://www.cplusplus.com/reference/algorithm/}
\item
  Ce sont des fonctions génériques qui prennent en argument des
  conteneurs (ou du moins des itérateurs) et des foncteurs

  \begin{itemize}
  \itemsep1pt\parskip0pt\parsep0pt
  \item
    elles appliquent les foncteurs à chaque élément du conteneur (entre
    les itérateurs de début et de fin)
  \end{itemize}
\item
  Objectif : profiter à la fois de la puissance de la STL et de la
  qualité des algorithmes fournis (exemple : tri)
\item Pas moins de 85 algorithmes sont proposés par la STL (et ce sont tous des \textit{templates})
\item Les algorithmes s'appliquent à tous les conteneurs
\begin{itemize}
\item Aux restrictions sur les types d'itérateurs près
\end{itemize}
\end{itemize}
\end{frame}

\begin{frame}[fragile]\frametitle{Remplissage d'un vecteur}
\begin{itemize}
\item Rappel : nous avons conçu un foncteur \verb|myfill|
\item Son application donnait lieu à une boucle, limitant son intérêt
\begin{codeblock}{Version sans algorithme}
\begin{lstlisting}
    myfill f(22);
    for (vector<int>::iterator i=v1.begin() ; i!=v1.end() ; ++i) {
        *i = f();
    }
\end{lstlisting}
\end{codeblock}
\item Nouvelle version avec l'utilisation de l'algorithme \verb|generate|
\begin{codeblock}{Version avec algorithme}
\begin{lstlisting}
    myfill f1(22);
    generate(v1.begin(),v1.end(),f1);
\end{lstlisting}
\end{codeblock}
\item Utilisation des itérateurs obligatoire (plus puissant)
\begin{itemize}
\item \textbf{Tous les algorithmes utiliseront des itérateurs plutôt que des conteneurs}
\end{itemize}
\end{itemize}
\end{frame}

\begin{frame}[fragile]\frametitle{Remplissage aléatoire et comptage}
\begin{itemize}
\item On veut écrire un code qui remplit un vecteur de façon aléatoire et qui compte le nombre de fois où apparait un nombre donné
%\item Foncteur de remplissage aléatoire
\pause\begin{codeblock}{classe \texttt{randomfill}}
\begin{lstlisting}
class randomfill {
private:
    int max;
public:
    randomfill(int m) : max(m) {}
    int operator()() {
        return rand()%max;
    }
};
\end{lstlisting}
\end{codeblock}
%\item Utilisation des algorithmes \verb|generate| et \verb|count|
\pause\begin{codeblock}{Code d'utilisation}
\begin{lstlisting}
    vector<int> v2(100);
    randomfill rf(15);
    generate(v2.begin(),v2.end(),rf);
    cout << "v2 contient " << count(v2.begin(),v2.end(),2) << " fois le nombre 2" << endl;
\end{lstlisting}
\end{codeblock}
%\begin{block}{Résultat}
%{\tiny
%\begin{verbatim}
%v2 contient 6 fois le nombre 2
%\end{verbatim}
%}
%\end{block}
\end{itemize}
\end{frame}

\begin{frame}[fragile]\frametitle{Utilisation des prédicats}
\begin{itemize}
\item Écrire un foncteur qui vérifie si une chaîne de caractère contient une voyelle. Compter combien de chaines d'un conteneur contiennent une voyelle avec  \verb|count_if| \urldoc{http://www.cplusplus.com/reference/algorithm/count_if/}
\end{itemize}
\pause\begin{codeblock}{Foncteur pour vérifier si un caractère est une voyelle}
\begin{lstlisting}
class isvowel {
public:
    bool operator()(char c) {
        if (c == 'a' || c == 'e' || c == 'y' || c == 'u' || c == 'i' || c == 'o') {
            return true; }
        else { return false; }
    }
};
\end{lstlisting}
\end{codeblock}
\pause\begin{codeblock}{Foncteur pour vérifier si une chaine contient une voyelle}
\begin{lstlisting}
class containsvowel {
public:
    bool operator()(string const &s) {
        return any_of(s.begin(), s.end(), isvowel());
    }
};
\end{lstlisting}
\end{codeblock}

\end{frame}

\begin{frame}[fragile]\frametitle{Utilisation des prédicats}
\begin{codeblock}{Test du foncteur \texttt{containsvowel}}
\begin{lstlisting}
    containsvowel cv;
    cout << (cv("ghjg") ? "voyelle" : "pas de voyelle") << endl;
    cout << (cv("fe(rgfgt") ? "voyelle" : "pas de voyelle") << endl;
\end{lstlisting}
\end{codeblock}
\pause
\begin{block}{Résultat}
{\tiny
\begin{verbatim}
pas de voyelle
voyelle
\end{verbatim}}
\end{block}
\pause\begin{codeblock}{Application à un conteneur}
\begin{lstlisting}
    list<string> ls;
    ls.push_back("kfkgike");
    ls.push_back("jfjgg");
    ls.push_back("i");
    ls.push_back("ayrud");
    cout << "nb chaine avec voyelle : " << count_if(ls.begin(),ls.end(),cv);
\end{lstlisting}
\end{codeblock}
\pause
\begin{block}{Résultat}
{\tiny
\begin{verbatim}
3
\end{verbatim}}
\end{block}
\end{frame}

\begin{frame}[fragile]\frametitle{Autres algorithmes d'utilisation courante (1/2)}
\begin{itemize}
\item Au delà des opérations déjà contenues dans les conteneurs, on trouve des prédicats pour les opérations courantes
\item Recherche
\begin{itemize}
\item \texttt{find} \urldoc{http://www.cplusplus.com/reference/algorithm/find/} pour faire une recherche (s'utilise comme \texttt{count}))
\item \verb|find_if| \urldoc{http://www.cplusplus.com/reference/algorithm/find_if/} qui permet d'utiliser un foncteur qui précise ce qu'on cherche
\end{itemize}
\item Minimum et maximum
\begin{itemize}
\item \verb|min_element| \urldoc{http://www.cplusplus.com/reference/algorithm/min_element/} permet de trouver le minimum d'un conteneur
\end{itemize}
\item Tri
\begin{itemize}
\item \verb|sort| \urldoc{http://www.cplusplus.com/reference/algorithm/sort/} pour un tri sur des conteneurs avec \textit{random access iterators}, utilise l'opérateur \verb|<| ou un comparateur dédié si utilisé avec 3 arguments
\item \verb|stable_sort| \urldoc{http://www.cplusplus.com/reference/algorithm/stable_sort/} pour un tri stable
\item \verb|is_sorted| \urldoc{http://www.cplusplus.com/reference/algorithm/is_sorted/} qui vérifie si un conteneur est trié
\end{itemize}
\end{itemize}
\end{frame}

% il faudrait mentionner les merger et le foreach
\begin{frame}[fragile]\frametitle{Autres algorithmes d'utilisation courante (2/2)}
\begin{itemize}
\item Application d'une fonction à un conteneur : \verb|for_each| \urldoc{http://www.cplusplus.com/reference/algorithm/for_each/}
\begin{itemize}
\item \verb|for_each| a l'avantage de s'utiliser avec un foncteur ou une simple fonction
\end{itemize}
\item \verb|transform| \urldoc{http://www.cplusplus.com/reference/algorithm/transform/} qui permet de travailler sur deux conteneurs à la fois (par exemple pour additionner des vecteurs terme à terme)
\begin{codeblock}{Exemple}
\begin{lstlisting}
    vector<int> a(50),b(50),c(50);
    generate(a.begin(), a.end(), randomfill(10));
    generate(b.begin(),b.end(), randomfill(10));
    transform(a.begin(),a.end(),b.begin(),c.begin(),plus<int>());
\end{lstlisting}
\end{codeblock}
\begin{itemize}
\item Remarque : on peut utiliser des objets sans les nommer : \verb|plus<int>()|
\end{itemize}
\end{itemize}
\end{frame}

\subsection{Les flux}

\begin{frame}{Introduction}
\begin{itemize}
\item Les flux (flots) sont utilisés par C++ pour les entrées/sorties, que ce soit sous forme de fichier, d'accès réseau ou d'interaction avec l'utilisateur (écran, clavier)
\item Depuis la première année, on manipule un outil bizarre : \texttt{cout} (et \texttt{cin})
\item Rien à voir (à première vue) : il existe différences catégories d'itérateurs (accès aléatoires, bidirectionnels...)
\item Un flux (flot de données) est une séquence sur laquelle on ne peut qu'avancer (i.e. sur laquelle un itérateur sera unidirectionnel)
\item De plus, il y aura certaines restrictions en lecture/écriture sur les itérateurs de flux
\end{itemize}
\end{frame}

\begin{frame}[fragile]{\texttt{cout} et itérateurs}
\begin{codeblock}{Etudions ce morceau de code}
\begin{lstlisting}
#include <iostream>
#include <iterator>
using namespace std;

int main() {
    ostream_iterator<double> it(cout);
    *it = 22.5;
    *it = 2.454;

}
\end{lstlisting}
\end{codeblock}
\begin{block}{Résultat}
{\tiny
\begin{verbatim}
22.52.454[mac-infmat05:optionRV/CPLUS/code] moreau%
\end{verbatim}
}
\end{block}
\begin{itemize}
\item des \texttt{include} sans surprise
\item notation curieuse pour les itérateurs (sans \verb|::|)
\item présence du type paramétrique \textit{a posteriori}
\end{itemize}
\end{frame}

\begin{frame}[fragile]{\texttt{cout} et itérateurs}
\begin{itemize}
\item Il manque un espace. Le 2ème argument précise le délimiteur
\begin{lstlisting}
    ostream_iterator<double> it(cout," ");
\end{lstlisting}
\item Si on utilise \verb|ofstream| au lieu de \verb|ostream|, on peut écrire dans un fichier
\item Plus intéressant, la lecture dans un fichier se fait avec \verb|ifstream|
\begin{itemize}
\item NB : on pourrait se passer des itérateurs (pour l'instant)
\end{itemize}
\begin{codeblock}{Lecture dans un fichier}
\begin{lstlisting}
    ifstream inf("data.txt");
    istream_iterator<double> itf(inf);
    istream_iterator<double> the_end;

    int cpt = 0;
    while (itf != the_end) {
        cout << "valeur numero " << cpt << " : " << *itf << endl;
        *itf++; cpt++;
    }
\end{lstlisting}
\end{codeblock}
\end{itemize}
\end{frame}

\begin{frame}[fragile]\frametitle{Le retour des algorithmes}
\begin{itemize}
\item Cas d'utilisation typique : lecture d'un fichier de valeurs réelles sur lequel on va effectuer des traitements statistiques
\item Utilisation de l'algorithme \verb|copy|
\begin{codeblock}{Lecture d'un vecteur sans boucle}
\begin{lstlisting}
    vector<double> v(10);
    ifstream inf2("data.txt");
    istream_iterator<double> itf2(inf2);
    copy(itf2,the_end,v.begin());
    copy(v.begin(),v.end(),ostream_iterator<double>(cout,"--"));
\end{lstlisting}
\end{codeblock}
\pause\begin{block}{Résultat}
{\tiny
\begin{verbatim}
12.4--24.45--23.455--33.22--45.556--0--0--0--0--0--
\end{verbatim}
}
\end{block}
\pause
\item Problème : le vecteur est pré-dimensionné
\item Solution : remplacer l'itérateur par un \verb|back_insert_iterator<vector<double> >|
\end{itemize}
\end{frame}

\begin{frame}[fragile]\frametitle{Flux et algorithmes}
\begin{codeblock}{Nouvelle version}
\begin{lstlisting}
    vector<double> v;
    ifstream inf2("data.txt");
    istream_iterator<double> itf2(inf2);
    back_insert_iterator<vector<double> > it2(v);
    copy(itf2,the_end,it2);
    copy(v.begin(),v.end(),ostream_iterator<double>(cout,"--"));
\end{lstlisting}
\end{codeblock}
\pause\begin{block}{Résultat}
{\tiny
\begin{verbatim}
12.4--24.45--23.455--33.22--45.556--
12.4--23.455--24.45--33.22--45.556--
\end{verbatim}
}
\end{block}
\begin{itemize}
\item On peut utiliser le mécanisme des algorithmes sur les itérateurs de flux \textbf{sans} charger tout un fichier en mémoire
\item Exercice : donner le max d'un fichier de réels en \textbf{deux} lignes de code !
\end{itemize}
\begin{codeblock}{Réponse}
\begin{lstlisting}
    ifstream if3("data.txt");
    cout << *max_element(istream_iterator<double>(if3),the_end) << endl;
\end{lstlisting}
\end{codeblock}
\end{frame}

\subsection{Cas particulier : flux et chaines de caractères}

\begin{frame}[fragile]
\frametitle{Flux et itérateurs de chaines de caractères}
\begin{itemize}
\item Jusque là, on s'est contenté de \verb|[]| et de \verb|+| pour manipuler les chaines
\item Flux chaines de caractères : \verb|ostringstream| et \verb|istringstream|
\begin{itemize}
\item On récupère la chaine construite avec la méthode \verb|str()|
\end{itemize}
\begin{codeblock}{Exemple}
\begin{lstlisting}
    ostringstream fl;
    fl << "une chaine " << 3.14 ;
\end{lstlisting}
\end{codeblock}
\item Les chaines contiennent également des itérateurs
%\begin{codeblock}{Foncteur de mise en majuscule}
%\begin{lstlisting}
%class toupperfct {
%public:
%    char operator()(char c) const {
%        return toupper(c);
%    }
%};
%\end{lstlisting}
%\end{codeblock}
\begin{codeblock}{Mise en majuscules d'une chaine de caractères}
\begin{lstlisting}
class toupperfct {
public:
    char operator()(char c) const {
        return toupper(c);
    }
};
int main() {
    string ch("Ceci est une chaine");
    transform(ch.begin(), ch.end(), ch.begin(), toupperfct());
    cout << ch << endl;
}
\end{lstlisting}
\end{codeblock}
\end{itemize}
\end{frame}


\section{C++ avancé}

% tout ce qui sort un peu du cadre

% tout ce qui concerne les templates

\subsection{Les templates}

\begin{frame}[t]{Les templates : motivation}
  \begin{itemize}
    \item Rappel : l'ordinateur est là pour simplifier les tâches répétitives
    \begin{itemize}
      \item Cela inclut la programmation elle-même !
    \end{itemize}
    \item Source d'inspiration : la classe \texttt{vector<?>}
    \begin{itemize}
      \item Elle fonctionne sur tout type de données
      \item Elle contient des \emph{vrais} types, pas des \texttt{void *}
      \item Limite les risques d'erreur de manipulation des vecteurs
    \end{itemize}
    \item On a croisé ces \texttt{<?>} un peu partout dans la bibliothèque standard
  \end{itemize}
\end{frame}
%--- Next Frame ---%


\begin{frame}[fragile]\frametitle{Un premier exemple simple}
  \begin{itemize}
    \item On a une fonction classique qui calcule le max de deux entiers
    \end{itemize}
\begin{lstlisting}
  int maximum(int a, int b) {
  if (a>b) {
    return a;
  }
  else {
    return b;
  }
}
  \end{lstlisting}
  \begin{itemize}
\item On aimerait pouvoir l'appliquer à tout type de données, sans avoir à réécrire la fonction
\item A supposer qu'on dispose d'un opérateur \texttt{<} sur le type en question
  \end{itemize}
\end{frame}

%--- Next Frame ---%

\begin{frame}[fragile]\frametitle{Fonctions template}
  \begin{itemize}
    \item Pour créer ces fonctions génériques, on va considérer qu'elles travaillent sur un type abstrait
    \item type abstrait ici = type pas défini au moment où on écrit le code de la fonction
    \item On l'appelle souvent \texttt{T}
    \item On préfixe alors la déclaration de fonction par \texttt{template <typename T>}
    \item On utilise T comme type générique dans la fonction
  \end{itemize}
  \begin{lstlisting}
    template<typename T> T maximumTemplate(const T& a,const T& b) {
      if (a>b) {
        return a;
      }
      else {
        return b;
      }
    }

  \end{lstlisting}
\end{frame}
% %--- Next Frame ---%


\begin{frame}[fragile]\frametitle{Remarques sur l'utilisation}
  \begin{itemize}
    \item Il est recommandé de spécifier le type sur lequel on veut instancier la fonction
    \begin{lstlisting}
      cout << maximumTemplate<double>(2.4,6.5) << endl;
      cout << maximumTemplate<int>(2,4) << endl;
    \end{lstlisting}
    \item Mais ce n'est pas obligatoire
    \begin{lstlisting}
      cout << maximumTemplate(2.4,6.5) << endl;
    \end{lstlisting}
      \item Attention aux ambiguités
      \begin{lstlisting}
        cout << maximumTemplate(2.4,6) << endl;
      \end{lstlisting}
  \end{itemize}
  {\tiny \begin{verbatim}
    maximum.cpp:30:34: warning: implicit conversion from 'double' to 'int' changes value from 2.5 to 2
      [-Wliteral-conversion]
    cout << maximumTemplate<int>(2.5,4) << endl;
            ~~~~~~~~~~~~~~~      ^~~
  \end{verbatim}}
\end{frame}
%--- Next Frame ---%

\begin{frame}[fragile]\frametitle{Précautions d'usage}
  \begin{itemize}
    \item L'utilisation des templates est encore difficile avec certains compilateurs
    \item Conséquences
    \begin{itemize}
      \item Dans la plupart des cas, il faut tout caser dans les .h
      \item Les messages d'erreurs des compilateurs peuvent être complexes à comprendre
    \end{itemize}
    \item Exemple : si vous utilisez un opérateur surchargé (ne serait-ce que \texttt{=}), tous les types \texttt{T} avec lesquels vous emploierez la fonction devront surcharger \texttt{=}
    \begin{itemize}
      \item constructeur par recopie indispensable
    \end{itemize}
    \item Rien ne garantit comment sera utilisée votre fonction template
    \begin{lstlisting}
      maximumTemplate(string("titi"),string("toto")
      maximumTemplate<Elephant>(...,...)
    \end{lstlisting}
  \end{itemize}
\end{frame}
%--- Next Frame ---%

\begin{frame}[fragile]{De la généricité ... mais pas trop !}
  \begin{itemize}
    \item On peut spécialiser les fonctions template (comportement particulier pour une classe spécifique) !
    \item Il suffit de préciser le type sur lequel elle s'applique
    \begin{lstlisting}
      template<> Elephant maximumTemplate(const Elephant&a,const Elephant&b) {
          if (a.getTaille() > b.getTaille()) {
              return a;
          }
          else {
              return b;
          }
    \end{lstlisting}
    \item utilisation
    \begin{lstlisting}
      maximumTemplate<Elephant>(e1,e2);
    \end{lstlisting}
    \item Attention à l'ordre de compilation : du plus générique au plus particulier
  \end{itemize}
\end{frame}
%--- Next Frame ---%

\begin{frame}[fragile]\frametitle{Classes template}
  \begin{itemize}
    \item Les types paramétriques ne sont pas réservés aux fonctions !
    \item Exemple : dans la classe \texttt{point}, on peut imaginer avoir besoin de points à coordonnées réelles ou entières
    \item Pas besoin de faire deux classes !
    \item On fait alors précéder la déclaration de la classe de \texttt{template<typename T>} où \texttt{T} est le type paramétrique
    \item Exemple
    \begin{lstlisting}
      template<typename T> class point {
      protected:
        T x;
        T y;

      public:
        point(T x,T y) {
          this->x = x;
          this->y = y;
        }

        // ...

        T getX() const {
          return this->x;
        }
      };
    \end{lstlisting}
  \end{itemize}
\end{frame}
%--- Next Frame ---%

 %\begin{frame}[fragile]\frametitle{Quelques compléments}
 %  \begin{itemize}
 %    \item Méthode de classe déclarée en dehors de la classe
%     \begin{itemize}
%       \item Déclaration standard dans la classe
%       \begin{lstlisting}
%           T norm2() const ;
%       \end{lstlisting}
%       \item Pour la définition, on ajoute le \texttt{template<typename T>} mais aussi le type de la classe
%       \begin{lstlisting}
%         template<typename T> T point<T>::norm2() const {
%           return x*x+y*y;
%         }
%       \end{lstlisting}
%     \end{itemize}
%  \item utilisation d'une classe template
 %  \end{itemize}
  % \end{frame}

\begin{frame}[fragile]\frametitle{Quelques compléments}
\begin{itemize}
	\item Méthode en dehors de la classe
	\begin{itemize}
		\item Déclaration seule dans la classe
\begin{lstlisting}
T norm2() const ;
\end{lstlisting}
		\item Code en dehors de la classe
		\begin{enumerate}
			\item ajout du \texttt{template<typename T>}
			\item ajout du type paramétrique après le nom de classe
		\end{enumerate}
		\begin{lstlisting}
		       template<typename T> T point<T>::norm2() const {
		                  return x*x+y*y;
		       }
		\end{lstlisting}
	\end{itemize}
	\item Utilisation
	\begin{lstlisting}
	  point<double> p1;
	  point<int> p2(2,2);
      cout << p2.norm2() << endl;
	\end{lstlisting}
\end{itemize}
\end{frame}


\subsection{Les conversions de type}

\begin{frame}[fragile]\frametitle{Conversion de type en C}
\begin{itemize}
\item En C, on utilise l'opérateur \texttt{(} \textit{type} \texttt{)} pour effectuer une conversion d'un type vers un autre
\begin{lstlisting}
int z = 12;
float a = (float) z;
ClasseQuelconque *q = new ClasseDerivee();
ClasseDerivee *r = (ClasseDerivee *) q; // downcasting
\end{lstlisting}
\item On a vu que cela pouvait être dangereux
\begin{lstlisting}
    int a[10];
    float *b;
    b = (float *)((void *) a);
\end{lstlisting}
\item C++ distingue 4(+1) types de conversion
\begin{itemize}
\item \textit{la conversion implicite}, i.e. \verb|short a = 2000; int b =a;|
\item La conversion \textbf{statique} de types
\item La \textbf{ré-interprétation} de données vers un autre type
\item la \textit{conversion} d'un objet \textbf{constant} vers un objet non constant
\item la conversion \textbf{dynamique} de types
\end{itemize}
\end{itemize}
\end{frame}

\begin{frame}{La conversion implicite}
\begin{itemize}
\item C'est ce qui se passe quand on ne précise rien
\item \structure{Promotion} : d'un type vers un type plus large, i.e. qui inclut le type d'origine
\begin{itemize}
\item de \texttt{short} vers \texttt{int}, de \texttt{float} vers \texttt{double}
\end{itemize}
\item Dans l'autre sens, cela \underline{induit une perte de précision}
\begin{itemize}
\item Cela devrait générer un avertissement de la part du compilateur (java génère même une erreur)
\item Cet avertissement peut être évité en faisant une conversion explicite
\end{itemize}
\item Règles générales
\begin{itemize}
\item  pointeur \textit{null} $\longrightarrow$ tout type de pointeur
\item  pointeur de tous types $\longrightarrow$ pointeur void*
\item \textit{upcast} : un pointeur sur une classe dérivée peut être converti vers un pointeur d'une classe dont il hérite
\end{itemize}
\end{itemize}
\end{frame}
\subsubsection{Opérateur \texttt{static\_cast}}

\begin{frame}[fragile]
\frametitle{\texttt{static\_cast}}
\begin{itemize}
\item Syntaxe : \textit{type} \textit{id} \verb|= static_cast<|\textit{type}\verb|>|\verb|(|\textit{exp}\verb|)|
\item Exemple
\begin{lstlisting}
    double d = 561.6516516;
    float f1 = static_cast<float>(d);

    double * ptr1 = new double;
    void * ptr2 = ptr1;
    double *ptr3 = static_cast<double*>(ptr2);
    double *ptr4 = ptr2;
\end{lstlisting}
\begin{block}{Erreur de compilation}
{\tiny
\begin{verbatim}
cast.cxx:10:13: error: cannot initialize a variable of type 'double *' with an lvalue of type 'void *'
    double *ptr4 = ptr2;
            ^      ~~~~
\end{verbatim}
}
\end{block}
\item \textcolor{red}{Attention} : \verb|static_cast| ne s'utilise que pour des types compatibles, qui n'impliquent pas de ré-interprétation des données
\begin{itemize}
\item Pas question de convertir un \verb|double *| vers un \verb|float *|
\item objets : s'applique au surclassement et au sousclassement (si possible)
\end{itemize}
\end{itemize}
\end{frame}

\subsubsection{Opérateur \texttt{reinterpret\_cast}}

\begin{frame}[fragile]
\frametitle{\texttt{reinterpret\_cast}}
\begin{itemize}
\item \verb|reinterpret_cast| fait la même chose que \verb|static_cast| mais y ajoutant une \struc{ré-interprétation} \textit{a posteriori} des données
\item Tous les types d'origine et de destination sont autorisés, mais ce n'est valable qu'à la compilation
\item En réalité \verb|reinterpret_cast| fait une copie binaire d'un type vers l'autre
\begin{itemize}
\item Aucune garantie, suppose la compatibilité binaire des deux classes
\item Absolument pas portable
\end{itemize}
\begin{codeblock}{Exemple "théorique"}
\begin{lstlisting}
class A { ... };
class B { ... };

A *a = new A;
B *b = reinterpret_cast<B*>(a);
\end{lstlisting}
\end{codeblock}
\end{itemize}
\end{frame}

\subsubsection{Opérateur \texttt{const\_cast}}

\begin{frame}[fragile]
\frametitle{\texttt{const\_cast}}
\begin{itemize}
\item \verb|const_cast| sert à modifier des valeurs déclarées \verb|const| !
\item Son utilisation est \alert{plus que dangeureuse} et doit donc être extrêmement réduite !
\begin{codeblock}{Exemple "convenable"}
\begin{lstlisting}
void print (char * str)
{
  cout << str << '\n';
}

int main () {
  const char * c = "sample text";
  print ( const_cast<char *> (c) );
  return 0;
}
\end{lstlisting}
\end{codeblock}
\pause \item Il serait néanmoins plus convenable de déclarer \verb|print() const|
\end{itemize}
\end{frame}

\subsubsection{Opérateur \texttt{dynamic\_cast}}

\begin{frame}[fragile]
\frametitle{\texttt{dynamic\_cast}}
\begin{itemize}
\item L'opérateur le plus utile avec \verb|static_cast| : il est utilisé pour les pointeurs (et les références) sur des objets
\item Son utilisation est interdite avec les autres types de données
\item Utilisation : sur et sous-classement des pointeurs vers des objets en vue de la mise en \oe uvre du polymorphisme
\item Intérêt pour le sousclassement : si RTTI est activé, le sousclassement ne sera effectué que si la conversion est valide
\begin{itemize}
\item en cas d'échec renvoie \verb|null| pour un pointeur
\item en cas d'échec lève une exception \verb|bad_cast| pour une référence
\end{itemize}
\end{itemize}
\end{frame}

\begin{frame}[fragile]
\frametitle{Exemple d'utilisation de \texttt{dynamic\_cast}}
\begin{columns}
\begin{column}{.48\textwidth}
\begin{codeblock}{classe \texttt{quelconque}}
\begin{lstlisting}
class quelconque {
public:
    virtual void bidon() { }
};
\end{lstlisting}
\end{codeblock}
\end{column}
\begin{column}{.48\textwidth}
\begin{codeblock}{classe \texttt{derivee}}
\begin{lstlisting}
class derivee : public quelconque {
public:
    void bidon() { }
};
\end{lstlisting}
\end{codeblock}
\end{column}
\end{columns}
\begin{codeblock}{Tests}
\begin{lstlisting}
    try {
        quelconque * pa = new derivee;
        quelconque * pb = new quelconque;
        derivee * pc;
        pc = dynamic_cast<derivee*>(pa);
        if (pc == NULL) {
            cout << "echec 1" << endl;
        }
        pc = dynamic_cast<derivee*>(pb);
        if (pc == NULL) {
            cout << "echec 2" << endl;
        }
        *pc = dynamic_cast<derivee&>(*pb);
    } catch(exception const &e) {
        cout << "exception: " << e.what() << endl;
    }
\end{lstlisting}
\end{codeblock}
\end{frame}

\subsection{l'introspection}
\label{sec:typeid}

\begin{frame}[fragile]\frametitle{Introspection en C++ : \texttt{typeid}}
\begin{itemize}
\item Certains langages permettent l'introspection ou la réflexivité : c'est la capacité à examiner ses structures de données internes, à examiner son propre état (voire à le modifier : introspection)
\item Elle est assez limitée en C++ (sous-entendu par rapport à java)
\item Utilisation du mot-clé \verb|typeid|
\begin{itemize}
\item \verb|typeid(|\textit{expression}\verb|)| renvoie un objet constant de type \verb|type_info|
\item Les valeurs de \verb|type_info| sont comparables via l'opérateur \verb|==|
\item On accède au nom d'une classe en utilisant la méthode \verb|type_info::name()|
\end{itemize}
\end{itemize}
\end{frame}

\begin{frame}[fragile]\frametitle{Exemple}
\begin{codeblock}{Code}
\begin{lstlisting}
    try {
        quelconque *a = new quelconque;
        quelconque *b = new derivee;

        cout << "a is: " << typeid(a).name() << endl;
        cout << "b is: " << typeid(b).name() << endl;
        cout << "*a is: " << typeid(*a).name() << endl;
        cout << "*b is: " << typeid(*b).name() << endl;
    } catch (exception& e) { cout << "Exception: " << e.what() << endl; }
\end{lstlisting}
\end{codeblock}
\begin{block}{Résultat (dépendant du compilateur utilisé)}
{\tiny
\begin{verbatim}
a is: P10quelconque
b is: P10quelconque
*a is: 10quelconque
*b is: 7derivee
\end{verbatim}
}
\end{block}
\end{frame}




\section{Bibliographie sommaire}

\begin{frame}{Bibliographie}
\begin{itemize}
\item B. Stroustrup (2013). The C++ programming Language, 4th edition. Addison Wesley.
\end{itemize}
\end{frame}

\begin{frame}{Cours en ligne}
\begin{itemize}
\item OpenClassrooms \url{http://fr.openclassrooms.com/informatique/cours/programmez-avec-le-langage-c}
\item cplusplus.com \url{http://www.cplusplus.com/doc/tutorial/}
\end{itemize}
\end{frame}
\end{document}
