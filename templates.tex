% tout ce qui concerne les templates

\subsection{Les templates}

\begin{frame}[t]{Les templates : motivation}
  \begin{itemize}
    \item Rappel : l'ordinateur est là pour simplifier les tâches répétitives
    \begin{itemize}
      \item Cela inclut la programmation elle-même !
    \end{itemize}
    \item Source d'inspiration : la classe \texttt{vector<?>}
    \begin{itemize}
      \item Elle fonctionne sur tout type de données
      \item Elle contient des \emph{vrais} types, pas des \texttt{void *}
      \item Limite les risques d'erreur de manipulation des vecteurs
    \end{itemize}
    \item On a croisé ces \texttt{<?>} un peu partout dans la bibliothèque standard
  \end{itemize}
\end{frame}
%--- Next Frame ---%


\begin{frame}[fragile]\frametitle{Un premier exemple simple}
  \begin{itemize}
    \item On a une fonction classique qui calcule le max de deux entiers
    \end{itemize}
\begin{lstlisting}
  int maximum(int a, int b) {
  if (a>b) {
    return a;
  }
  else {
    return b;
  }
}
  \end{lstlisting}
  \begin{itemize}
\item On aimerait pouvoir l'appliquer à tout type de données, sans avoir à réécrire la fonction
\item A supposer qu'on dispose d'un opérateur \texttt{<} sur le type en question
  \end{itemize}
\end{frame}

%--- Next Frame ---%

\begin{frame}[fragile]\frametitle{Fonctions template}
  \begin{itemize}
    \item Pour créer ces fonctions génériques, on va considérer qu'elles travaillent sur un type abstrait
    \item type abstrait ici = type pas défini au moment où on écrit le code de la fonction
    \item On l'appelle souvent \texttt{T}
    \item On préfixe alors la déclaration de fonction par \texttt{template <typename T>}
    \item On utilise T comme type générique dans la fonction
  \end{itemize}
  \begin{lstlisting}
    template<typename T> T maximumTemplate(const T& a,const T& b) {
      if (a>b) {
        return a;
      }
      else {
        return b;
      }
    }

  \end{lstlisting}
\end{frame}
% %--- Next Frame ---%


\begin{frame}[fragile]\frametitle{Remarques sur l'utilisation}
  \begin{itemize}
    \item Il est recommandé de spécifier le type sur lequel on veut instancier la fonction
    \begin{lstlisting}
      cout << maximumTemplate<double>(2.4,6.5) << endl;
      cout << maximumTemplate<int>(2,4) << endl;
    \end{lstlisting}
    \item Mais ce n'est pas obligatoire
    \begin{lstlisting}
      cout << maximumTemplate(2.4,6.5) << endl;
    \end{lstlisting}
      \item Attention aux ambiguités
      \begin{lstlisting}
        cout << maximumTemplate(2.5,4) << endl;
      \end{lstlisting}
  \end{itemize}
  {\tiny \begin{verbatim}
    maximum.cpp:30:34: warning: implicit conversion from 'double' to 'int' changes value from 2.5 to 2
      [-Wliteral-conversion]
    cout << maximumTemplate<int>(2.5,4) << endl;
            ~~~~~~~~~~~~~~~      ^~~
  \end{verbatim}}
\end{frame}
%--- Next Frame ---%

\begin{frame}[fragile]\frametitle{Précautions d'usage}
  \begin{itemize}
    \item L'utilisation des templates est encore difficile avec certains compilateurs
    \item Conséquences
    \begin{itemize}
      \item Dans la plupart des cas, il faut tout caser dans les .h
      \item Les messages d'erreurs des compilateurs peuvent être complexes à comprendre
    \end{itemize}
    \item Exemple : si vous utilisez un opérateur surchargé (ne serait-ce que \texttt{=}), tous les types \texttt{T} avec lesquels vous emploierez la fonction devront surcharger \texttt{=}
    \begin{itemize}
      \item constructeur par recopie indispensable
    \end{itemize}
    \item Rien ne garantit comment sera utilisée votre fonction template
    \begin{lstlisting}
      maximumTemplate(string("titi"),string("toto")
      maximumTemplate<Elephant>(...,...)
    \end{lstlisting}
  \end{itemize}
\end{frame}
%--- Next Frame ---%

\begin{frame}[fragile]{De la généricité ... mais pas trop !}
  \begin{itemize}
    \item On peut spécialiser les fonctions template (comportement particulier pour une classe spécifique) !
    \item Il suffit de préciser le type sur lequel elle s'applique
    \begin{lstlisting}
      template<> Elephant maximumTemplate(const Elephant&a,const Elephant&b) {
          if (a.getTaille() > b.getTaille()) {
              return a;
          }
          else {
              return b;
          }
    \end{lstlisting}
    \item utilisation
    \begin{lstlisting}
      maximumTemplate<Elephant>(e1,e2);
    \end{lstlisting}
    \item Attention à l'ordre de compilation : du plus générique au plus particulier
  \end{itemize}
\end{frame}
%--- Next Frame ---%

\begin{frame}[fragile]\frametitle{Classes template}
  \begin{itemize}
    \item Les types paramétriques ne sont pas réservés aux fonctions !
    \item Exemple : dans la classe \texttt{point}, on peut imaginer avoir besoin de points à coordonnées réelles ou entières
    \item Pas besoin de faire deux classes !
    \item On fait alors précéder la déclaration de la classe de \texttt{template<typename T>} où \texttt{T} est le type paramétrique
    \item Exemple
    \begin{lstlisting}
      template<typename T> class point {
      protected:
        T x;
        T y;

      public:
        point(T x,T y) {
          this->x = x;
          this->y = y;
        }

        // ...

        T getX() const {
          return this->x;
        }
      };
    \end{lstlisting}
  \end{itemize}
\end{frame}
%--- Next Frame ---%

 %\begin{frame}[fragile]\frametitle{Quelques compléments}
 %  \begin{itemize}
 %    \item Méthode de classe déclarée en dehors de la classe
%     \begin{itemize}
%       \item Déclaration standard dans la classe
%       \begin{lstlisting}
%           T norm2() const ;
%       \end{lstlisting}
%       \item Pour la définition, on ajoute le \texttt{template<typename T>} mais aussi le type de la classe
%       \begin{lstlisting}
%         template<typename T> T point<T>::norm2() const {
%           return x*x+y*y;
%         }
%       \end{lstlisting}
%     \end{itemize}
%  \item utilisation d'une classe template
 %  \end{itemize}
  % \end{frame}

\begin{frame}[fragile]\frametitle{Quelques compléments}
\begin{itemize}
	\item Méthode en dehors de la classe
	\begin{itemize}
		\item Déclaration seule dans la classe
\begin{lstlisting}
T norm2() const ;
\end{lstlisting}
		\item Code en dehors de la classe
		\begin{enumerate}
			\item ajout du \texttt{template<typename T>}
			\item ajout du type paramétrique après le nom de classe
		\end{enumerate}
		\begin{lstlisting}
		       template<typename T> T point<T>::norm2() const {
		                  return x*x+y*y;
		       }
		\end{lstlisting}
	\end{itemize}
	\item Utilisation
	\begin{lstlisting}
	  point<double> p1;
	  point<int> p2(2,2);
    cout << p2.norm2() << endl;
	\end{lstlisting}
\end{itemize}
\end{frame}
