% tout ce qui sort un peu du cadre

% tout ce qui concerne les templates

\subsection{Les templates}

\begin{frame}[t]{Les templates : motivation}
  \begin{itemize}
    \item Rappel : l'ordinateur est là pour simplifier les tâches répétitives
    \begin{itemize}
      \item Cela inclut la programmation elle-même !
    \end{itemize}
    \item Source d'inspiration : la classe \texttt{vector<?>}
    \begin{itemize}
      \item Elle fonctionne sur tout type de données
      \item Elle contient des \emph{vrais} types, pas des \texttt{void *}
      \item Limite les risques d'erreur de manipulation des vecteurs
    \end{itemize}
    \item On a croisé ces \texttt{<?>} un peu partout dans la bibliothèque standard
  \end{itemize}
\end{frame}
%--- Next Frame ---%


\begin{frame}[fragile]\frametitle{Un premier exemple simple}
  \begin{itemize}
    \item On a une fonction classique qui calcule le max de deux entiers
    \end{itemize}
\begin{lstlisting}
  int maximum(int a, int b) {
  if (a>b) {
    return a;
  }
  else {
    return b;
  }
}
  \end{lstlisting}
  \begin{itemize}
\item On aimerait pouvoir l'appliquer à tout type de données, sans avoir à réécrire la fonction
\item A supposer qu'on dispose d'un opérateur \texttt{<} sur le type en question
  \end{itemize}
\end{frame}

%--- Next Frame ---%

\begin{frame}[fragile]\frametitle{Fonctions template}
  \begin{itemize}
    \item Pour créer ces fonctions génériques, on va considérer qu'elles travaillent sur un type abstrait
    \item type abstrait ici = type pas défini au moment où on écrit le code de la fonction
    \item On l'appelle souvent \texttt{T}
    \item On préfixe alors la déclaration de fonction par \texttt{template <typename T>}
    \item On utilise T comme type générique dans la fonction
  \end{itemize}
  \begin{lstlisting}
    template<typename T> T maximumTemplate(const T& a,const T& b) {
      if (a>b) {
        return a;
      }
      else {
        return b;
      }
    }

  \end{lstlisting}
\end{frame}
% %--- Next Frame ---%


\begin{frame}[fragile]\frametitle{Remarques sur l'utilisation}
  \begin{itemize}
    \item Il est recommandé de spécifier le type sur lequel on veut instancier la fonction
    \begin{lstlisting}
      cout << maximumTemplate<double>(2.4,6.5) << endl;
      cout << maximumTemplate<int>(2,4) << endl;
    \end{lstlisting}
    \item Mais ce n'est pas obligatoire
    \begin{lstlisting}
      cout << maximumTemplate(2.4,6.5) << endl;
    \end{lstlisting}
      \item Attention aux ambiguités
      \begin{lstlisting}
        cout << maximumTemplate(2.4,6) << endl;
      \end{lstlisting}
  \end{itemize}
  {\tiny \begin{verbatim}
    maximum.cpp:30:34: warning: implicit conversion from 'double' to 'int' changes value from 2.5 to 2
      [-Wliteral-conversion]
    cout << maximumTemplate<int>(2.5,4) << endl;
            ~~~~~~~~~~~~~~~      ^~~
  \end{verbatim}}
\end{frame}
%--- Next Frame ---%

\begin{frame}[fragile]\frametitle{Précautions d'usage}
  \begin{itemize}
    \item L'utilisation des templates est encore difficile avec certains compilateurs
    \item Conséquences
    \begin{itemize}
      \item Dans la plupart des cas, il faut tout caser dans les .h
      \item Les messages d'erreurs des compilateurs peuvent être complexes à comprendre
    \end{itemize}
    \item Exemple : si vous utilisez un opérateur surchargé (ne serait-ce que \texttt{=}), tous les types \texttt{T} avec lesquels vous emploierez la fonction devront surcharger \texttt{=}
    \begin{itemize}
      \item constructeur par recopie indispensable
    \end{itemize}
    \item Rien ne garantit comment sera utilisée votre fonction template
    \begin{lstlisting}
      maximumTemplate(string("titi"),string("toto")
      maximumTemplate<Elephant>(...,...)
    \end{lstlisting}
  \end{itemize}
\end{frame}
%--- Next Frame ---%

\begin{frame}[fragile]{De la généricité ... mais pas trop !}
  \begin{itemize}
    \item On peut spécialiser les fonctions template (comportement particulier pour une classe spécifique) !
    \item Il suffit de préciser le type sur lequel elle s'applique
    \begin{lstlisting}
      template<> Elephant maximumTemplate(const Elephant&a,const Elephant&b) {
          if (a.getTaille() > b.getTaille()) {
              return a;
          }
          else {
              return b;
          }
    \end{lstlisting}
    \item utilisation
    \begin{lstlisting}
      maximumTemplate<Elephant>(e1,e2);
    \end{lstlisting}
    \item Attention à l'ordre de compilation : du plus générique au plus particulier
  \end{itemize}
\end{frame}
%--- Next Frame ---%

\begin{frame}[fragile]\frametitle{Classes template}
  \begin{itemize}
    \item Les types paramétriques ne sont pas réservés aux fonctions !
    \item Exemple : dans la classe \texttt{point}, on peut imaginer avoir besoin de points à coordonnées réelles ou entières
    \item Pas besoin de faire deux classes !
    \item On fait alors précéder la déclaration de la classe de \texttt{template<typename T>} où \texttt{T} est le type paramétrique
    \item Exemple
    \begin{lstlisting}
      template<typename T> class point {
      protected:
        T x;
        T y;

      public:
        point(T x,T y) {
          this->x = x;
          this->y = y;
        }

        // ...

        T getX() const {
          return this->x;
        }
      };
    \end{lstlisting}
  \end{itemize}
\end{frame}
%--- Next Frame ---%

 %\begin{frame}[fragile]\frametitle{Quelques compléments}
 %  \begin{itemize}
 %    \item Méthode de classe déclarée en dehors de la classe
%     \begin{itemize}
%       \item Déclaration standard dans la classe
%       \begin{lstlisting}
%           T norm2() const ;
%       \end{lstlisting}
%       \item Pour la définition, on ajoute le \texttt{template<typename T>} mais aussi le type de la classe
%       \begin{lstlisting}
%         template<typename T> T point<T>::norm2() const {
%           return x*x+y*y;
%         }
%       \end{lstlisting}
%     \end{itemize}
%  \item utilisation d'une classe template
 %  \end{itemize}
  % \end{frame}

\begin{frame}[fragile]\frametitle{Quelques compléments}
\begin{itemize}
	\item Méthode en dehors de la classe
	\begin{itemize}
		\item Déclaration seule dans la classe
\begin{lstlisting}
T norm2() const ;
\end{lstlisting}
		\item Code en dehors de la classe
		\begin{enumerate}
			\item ajout du \texttt{template<typename T>}
			\item ajout du type paramétrique après le nom de classe
		\end{enumerate}
		\begin{lstlisting}
		       template<typename T> T point<T>::norm2() const {
		                  return x*x+y*y;
		       }
		\end{lstlisting}
	\end{itemize}
	\item Utilisation
	\begin{lstlisting}
	  point<double> p1;
	  point<int> p2(2,2);
      cout << p2.norm2() << endl;
	\end{lstlisting}
\end{itemize}
\end{frame}


\subsection{Les conversions de type}

\begin{frame}[fragile]\frametitle{Conversion de type en C}
\begin{itemize}
\item En C, on utilise l'opérateur \texttt{(} \textit{type} \texttt{)} pour effectuer une conversion d'un type vers un autre
\begin{lstlisting}
int z = 12;
float a = (float) z;
ClasseQuelconque *q = new ClasseDerivee();
ClasseDerivee *r = (ClasseDerivee *) q; // downcasting
\end{lstlisting}
\item On a vu que cela pouvait être dangereux
\begin{lstlisting}
    int a[10];
    float *b;
    b = (float *)((void *) a);
\end{lstlisting}
\item C++ distingue 4(+1) types de conversion
\begin{itemize}
\item \textit{la conversion implicite}, i.e. \verb|short a = 2000; int b =a;|
\item La conversion \textbf{statique} de types
\item La \textbf{ré-interprétation} de données vers un autre type
\item la \textit{conversion} d'un objet \textbf{constant} vers un objet non constant
\item la conversion \textbf{dynamique} de types
\end{itemize}
\end{itemize}
\end{frame}

\begin{frame}{La conversion implicite}
\begin{itemize}
\item C'est ce qui se passe quand on ne précise rien
\item \structure{Promotion} : d'un type vers un type plus large, i.e. qui inclut le type d'origine
\begin{itemize}
\item de \texttt{short} vers \texttt{int}, de \texttt{float} vers \texttt{double}
\end{itemize}
\item Dans l'autre sens, cela \underline{induit une perte de précision}
\begin{itemize}
\item Cela devrait générer un avertissement de la part du compilateur (java génère même une erreur)
\item Cet avertissement peut être évité en faisant une conversion explicite
\end{itemize}
\item Règles générales
\begin{itemize}
\item  pointeur \textit{null} $\longrightarrow$ tout type de pointeur
\item  pointeur de tous types $\longrightarrow$ pointeur void*
\item \textit{upcast} : un pointeur sur une classe dérivée peut être converti vers un pointeur d'une classe dont il hérite
\end{itemize}
\end{itemize}
\end{frame}
\subsubsection{Opérateur \texttt{static\_cast}}

\begin{frame}[fragile]
\frametitle{\texttt{static\_cast}}
\begin{itemize}
\item Syntaxe : \textit{type} \textit{id} \verb|= static_cast<|\textit{type}\verb|>|\verb|(|\textit{exp}\verb|)|
\item Exemple
\begin{lstlisting}
    double d = 561.6516516;
    float f1 = static_cast<float>(d);

    double * ptr1 = new double;
    void * ptr2 = ptr1;
    double *ptr3 = static_cast<double*>(ptr2);
    double *ptr4 = ptr2;
\end{lstlisting}
\begin{block}{Erreur de compilation}
{\tiny
\begin{verbatim}
cast.cxx:10:13: error: cannot initialize a variable of type 'double *' with an lvalue of type 'void *'
    double *ptr4 = ptr2;
            ^      ~~~~
\end{verbatim}
}
\end{block}
\item \textcolor{red}{Attention} : \verb|static_cast| ne s'utilise que pour des types compatibles, qui n'impliquent pas de ré-interprétation des données
\begin{itemize}
\item Pas question de convertir un \verb|double *| vers un \verb|float *|
\item objets : s'applique au surclassement et au sousclassement (si possible)
\end{itemize}
\end{itemize}
\end{frame}

\subsubsection{Opérateur \texttt{reinterpret\_cast}}

\begin{frame}[fragile]
\frametitle{\texttt{reinterpret\_cast}}
\begin{itemize}
\item \verb|reinterpret_cast| fait la même chose que \verb|static_cast| mais y ajoutant une \struc{ré-interprétation} \textit{a posteriori} des données
\item Tous les types d'origine et de destination sont autorisés, mais ce n'est valable qu'à la compilation
\item En réalité \verb|reinterpret_cast| fait une copie binaire d'un type vers l'autre
\begin{itemize}
\item Aucune garantie, suppose la compatibilité binaire des deux classes
\item Absolument pas portable
\end{itemize}
\begin{codeblock}{Exemple "théorique"}
\begin{lstlisting}
class A { ... };
class B { ... };

A *a = new A;
B *b = reinterpret_cast<B*>(a);
\end{lstlisting}
\end{codeblock}
\end{itemize}
\end{frame}

\subsubsection{Opérateur \texttt{const\_cast}}

\begin{frame}[fragile]
\frametitle{\texttt{const\_cast}}
\begin{itemize}
\item \verb|const_cast| sert à modifier des valeurs déclarées \verb|const| !
\item Son utilisation est \alert{plus que dangeureuse} et doit donc être extrêmement réduite !
\begin{codeblock}{Exemple "convenable"}
\begin{lstlisting}
void print (char * str)
{
  cout << str << '\n';
}

int main () {
  const char * c = "sample text";
  print ( const_cast<char *> (c) );
  return 0;
}
\end{lstlisting}
\end{codeblock}
\pause \item Il serait néanmoins plus convenable de déclarer \verb|print() const|
\end{itemize}
\end{frame}

\subsubsection{Opérateur \texttt{dynamic\_cast}}

\begin{frame}[fragile]
\frametitle{\texttt{dynamic\_cast}}
\begin{itemize}
\item L'opérateur le plus utile avec \verb|static_cast| : il est utilisé pour les pointeurs (et les références) sur des objets
\item Son utilisation est interdite avec les autres types de données
\item Utilisation : sur et sous-classement des pointeurs vers des objets en vue de la mise en \oe uvre du polymorphisme
\item Intérêt pour le sousclassement : si RTTI est activé, le sousclassement ne sera effectué que si la conversion est valide
\begin{itemize}
\item en cas d'échec renvoie \verb|null| pour un pointeur
\item en cas d'échec lève une exception \verb|bad_cast| pour une référence
\end{itemize}
\end{itemize}
\end{frame}

\begin{frame}[fragile]
\frametitle{Exemple d'utilisation de \texttt{dynamic\_cast}}
\begin{columns}
\begin{column}{.48\textwidth}
\begin{codeblock}{classe \texttt{quelconque}}
\begin{lstlisting}
class quelconque {
public:
    virtual void bidon() { }
};
\end{lstlisting}
\end{codeblock}
\end{column}
\begin{column}{.48\textwidth}
\begin{codeblock}{classe \texttt{derivee}}
\begin{lstlisting}
class derivee : public quelconque {
public:
    void bidon() { }
};
\end{lstlisting}
\end{codeblock}
\end{column}
\end{columns}
\begin{codeblock}{Tests}
\begin{lstlisting}
    try {
        quelconque * pa = new derivee;
        quelconque * pb = new quelconque;
        derivee * pc;
        pc = dynamic_cast<derivee*>(pa);
        if (pc == NULL) {
            cout << "echec 1" << endl;
        }
        pc = dynamic_cast<derivee*>(pb);
        if (pc == NULL) {
            cout << "echec 2" << endl;
        }
        *pc = dynamic_cast<derivee&>(*pb);
    } catch(exception const &e) {
        cout << "exception: " << e.what() << endl;
    }
\end{lstlisting}
\end{codeblock}
\end{frame}

\subsection{l'introspection}
\label{sec:typeid}

\begin{frame}[fragile]\frametitle{Introspection en C++ : \texttt{typeid}}
\begin{itemize}
\item Certains langages permettent l'introspection ou la réflexivité : c'est la capacité à examiner ses structures de données internes, à examiner son propre état (voire à le modifier : introspection)
\item Elle est assez limitée en C++ (sous-entendu par rapport à java)
\item Utilisation du mot-clé \verb|typeid|
\begin{itemize}
\item \verb|typeid(|\textit{expression}\verb|)| renvoie un objet constant de type \verb|type_info|
\item Les valeurs de \verb|type_info| sont comparables via l'opérateur \verb|==|
\item On accède au nom d'une classe en utilisant la méthode \verb|type_info::name()|
\end{itemize}
\end{itemize}
\end{frame}

\begin{frame}[fragile]\frametitle{Exemple}
\begin{codeblock}{Code}
\begin{lstlisting}
    try {
        quelconque *a = new quelconque;
        quelconque *b = new derivee;

        cout << "a is: " << typeid(a).name() << endl;
        cout << "b is: " << typeid(b).name() << endl;
        cout << "*a is: " << typeid(*a).name() << endl;
        cout << "*b is: " << typeid(*b).name() << endl;
    } catch (exception& e) { cout << "Exception: " << e.what() << endl; }
\end{lstlisting}
\end{codeblock}
\begin{block}{Résultat (dépendant du compilateur utilisé)}
{\tiny
\begin{verbatim}
a is: P10quelconque
b is: P10quelconque
*a is: 10quelconque
*b is: 7derivee
\end{verbatim}
}
\end{block}
\end{frame}
