
\begin{frame}{Nouveautés}
    \begin{itemize}
        \item Chaque version de la norme C++ introduit des nouveautés ... plus ou moins implémentées par les compilateurs 
        \item On se focalise ici uniquement sur les plus directement utilisables (et les plus simples)
        \item Pour compiler en mode C++11, il faut ajouter l'option \texttt{-std=c++11} au compilateur
    \end{itemize}
\end{frame}

\begin{frame}[fragile]
\frametitle{Améliorations quant à la déclaration des types}
    \begin{itemize}
        \item \texttt{auto} : permet de ne pas déclarer le type d'une variable qui sera déduit par le compilateur 
        \item \texttt{decltype} : permet de déclarer le type d'une nouvelle variable en fonction du type d'une variable existante
        \item On peut aussi déclarer une fonction \texttt{decltype(auto)}, ou plus simplement \texttt{auto} en C++14
    \end{itemize}
\begin{lstlisting}
    vector<int> v1(20);

    for (auto it=v1.begin(); it != v1.end(); it++) {
        ...
    }

    decltype(v1) v2;

    // parfois etrange
    auto p = new auto(3.14159);
\end{lstlisting}
\end{frame}

\begin{frame}[fragile]
\frametitle{Instruction for in}
    \begin{itemize}
        \item Une nouvelle syntaxe pour l'instruction \texttt{for} a été introduite pour effectuer une itération sur un conteneur de la librairie standard 
        \item Elle existait déjà dans plusieurs compilateurs
    \end{itemize}
\begin{lstlisting}
vector<int> vect;

for (int v : vect) {
    sum += v;
}
\end{lstlisting}
\begin{itemize}
    \item qu'on peut alors écrire aussi 
\end{itemize}
\begin{lstlisting}
    for (auto v : vect) {
        sum += v;
    }
\end{lstlisting}
\end{frame}

\begin{frame}[fragile]
\frametitle{emplace\_back}
    \begin{itemize}
        \item \texttt{emplace\_back} permet de créer à la volée un élément et de l'insérer dans un vecteur par exemple 
        \item à utiliser à la place de \texttt{push\_back}
        \item Limite le nombre de copies et d'appel aux constructeurs
    \end{itemize}
\end{frame}

\begin{frame}[fragile]
\frametitle{Exemple}
\begin{lstlisting}
class Person1 {
public:
    string name;
    int age;
 
    Person1() {
        cout << "default constructor Person1" << endl;
        name = "unknown";
        age = -1;
    }
    Person1(string n, int a) : name(n), age(a) {
      cout << "construct Person1" << endl;
    } 
    Person1(const Person1& p) {
      cout << "copy constructor Person1" << endl;
      name = p.name;
      age = p.age;
    }      
    Person1& operator=(const Person1& p) {
      name = p.name;
      age = p.age;
      cout << "assign Person1" << endl;
      return *this;    
    }   
    friend ostream& operator<<(ostream& out, Person1& p) {
      out << p.name << ", " << p.age;
      return out;
    }
};
\end{lstlisting}
\end{frame}

\begin{frame}[fragile]
\frametitle{Exemple d'utilisation}
\begin{lstlisting}
void test1() {
    vector<Person1> v;
    cout << endl << "PUSH BACK" << endl;
    cout << string(40, '-') << endl;
    cout << "- push back" << endl;
    v.push_back(Person1("Masha", 5));
}
\end{lstlisting}
\begin{exampleblock}{Exécution}
\begin{verbatim}
    PUSH BACK
    ----------------------------------------
    - push back
    construct Person1
    copy constructor Person1    
\end{verbatim}
\end{exampleblock}
\end{frame}

\begin{frame}[fragile]
\frametitle{Avec emplace\_back}
\begin{itemize}
    \item Ajout d'un constructeur de type move
\end{itemize}
\begin{lstlisting}
    Person1(Person1&& p) : name(std::move(p.name)), age(p.age) {
        cout << "move Person1  { " << name << ", " << age << " }" << endl;
      }
\end{lstlisting}
\begin{itemize}
    \item Utilisation
\end{itemize}
\begin{lstlisting}
    void test2() {
        vector<Person1> v;
        cout << endl << "EMPLACE BACK" << endl; 
        cout << string(40, '-') << endl;
        v.emplace_back(Person1("Masha", 5));
}    
\end{lstlisting}
\begin{exampleblock}{Exécution}
\begin{verbatim}
    EMPLACE BACK
    ----------------------------------------
    construct Person1
    move Person1  { Masha, 5 }
\end{verbatim}
\end{exampleblock}
\end{frame}

\begin{frame}[fragile]
\frametitle{Smart pointers}
    \begin{itemize}
        \item La gestion des pointeurs et de la mémoire est le point faible de C(++), d'où l'introduction de nouveaux types 
        \item \texttt{unique\_ptr<T>} gère un pointeur de façon à ce qu'il n'y ait qu'une seule occurrence valide 
        \item \texttt{shared\_ptr<T>} gère plusieurs pointeurs sur un même espace mémoire et libère l'espace quand il n'est plus référencé (ressemble aux références java)
    \end{itemize}
\end{frame}

\begin{frame}[fragile]
\frametitle{unique\_ptr}
\begin{lstlisting}
    unique_ptr<int> p1(new int); 

    *p1 = 2;
    cout << "p1 = "  << *p1 << endl;

    // ATTENTION
    unique_ptr<int> p2(p1.get());
\end{lstlisting}
\begin{exampleblock}{Exécution}
    {\tiny}
    \begin{verbatim}        
    p1 = 2
    smart_pointers(96785,0x10d538dc0) malloc: *** error for object 0x7f9769405890: pointer being freed was not allocated
    smart_pointers(96785,0x10d538dc0) malloc: *** set a breakpoint in malloc_error_break to debug
    [1]    96785 abort      ./smart_pointers    
\end{verbatim}
\end{exampleblock}
\end{frame}

\begin{frame}[fragile]
\frametitle{unique\_ptr}
\begin{lstlisting}
    // move pour transferer a une nouvelle reference 
    unique_ptr<int> p2(std::move(p1)); 

    // on ne peut plus l'utiliser sous l'ancien nom 
     cout << "p1 = "  << *p1 << endl;
\end{lstlisting}
\begin{exampleblock}{Exécution}
{\tiny
\begin{verbatim}
[1]    2961 segmentation fault  ./smart_pointers
\end{verbatim}
}
\end{exampleblock}

\begin{lstlisting}
    *p2 = 7;
    cout << "p2 = " << *p2 << endl;
    // on rend la reference a p1
    p1 = std::move(p2);
    cout << "p1 = " << *p1 << endl;
\end{lstlisting}    
\end{frame}

\begin{frame}[fragile]
\frametitle{shared\_ptr}
\begin{lstlisting}
    // partie sur les smart pointers 
    shared_ptr<int> x1(new int);
    shared_ptr<int> x2(x1);
    shared_ptr<int> x3(x1);
 
    *x1 = 1;
    *x2 = 2;
    *x3 = 3;
    cout << "int is referenced " << x1.use_count() << " time(s)" << endl;
\end{lstlisting}  
\begin{exampleblock}{Exécution}
{\tiny
\begin{verbatim}
int is referenced 3 time(s)
\end{verbatim}
}
\end{exampleblock}
\begin{itemize}
    \item Permet de limiter les fuites de mémoire 
    \item Pas si facile à utiliser quand on l'utilise avec des shared pointers de pointeurs...
\end{itemize}
\end{frame}